\documentclass[twocolumn]{article}
\usepackage{mathpazo}
\usepackage{microtype}
\usepackage{amsmath}

\setlength\textwidth{7in} 
\setlength\textheight{9.5in} 
\setlength\oddsidemargin{-0.25in} 
\setlength\topmargin{-0.25in} 
\setlength\headheight{0in} 
\setlength\headsep{0in} 
\setlength\columnsep{18pt}
\sloppy 
 
\begin{document}

\title{
\vspace{-0.5in}\rule{\textwidth}{2pt}
\begin{tabular}{ll}\begin{minipage}{4.75in}\vspace{6px}
\noindent\LARGE Department of Computer Science\\
\vspace{-12px}\\
\noindent\large George Mason University\qquad Technical Reports
\end{minipage}&\begin{minipage}{2in}\vspace{6px}\small
4400 University Drive MS\#4A5\\
Fairfax, VA 22030-4444 USA\\
http:/$\!$/cs.gmu.edu/\quad 703-993-1530
\end{minipage}\end{tabular}
\rule{\textwidth}{2pt}\vspace{0.25in}
\LARGE \bf
Adaptive Bondsman and Robust Multiagent Task Allocation
}

\date{Project Proposal
March 4th 2015
}

\author{
{\bf Drew Wicke}\\
dwicke@gmu.edu
}

\maketitle

\begin{abstract}

The following document describes my intended project for CS 700.

\end{abstract}

\section{Introduction}
In Bounty Hunters and Multiagent Task Allocation we developed a new approach to dynamic multiagent task allocation.  The method described in the paper takes its inspiration from the United States version of bounty hunting where a bail bondsman can offer a bounty to bondsmen for the successful return of a fugitive that has skipped bail.  The bounties method offers a different approach to task allocation from the normal auction/market based approach that is well studied in the literature.  Bounties are attractive due to the non-exclusivity of the tasks, allowing more than one agent at a time to go after a task.  Bounties also do not use a bidding mechanism and since agents are not bidding their valuations of tasks can 

Unlike a typical auction, a true bounty system is not exclusive: multiple agents may commit to a task.  Exclusivity has an advantage in that agents are never wasting time simultaneously working on the same task.  But bounties have a different advantage: they provide a straightforward way for agents to resolve situations where an agent cannot complete a task or complete it efficiently, as  ultimately another agent will take it from him. 

\section{Problem Description}
In Bounty Hunters and Multiagent Task Allocation we showed that Bounties are more efficient in noisy situations than ``auction like'' models, but are slightly less efficient in more common scenarios.  Therefore, the problem is in developing a more general method that is efficient in both types of situations.  

My approach to this problem will be to improve the bondsman.  The bondsman is the entity that controls the bounty, start bounty and the exclusivity of the tasks.  In the original paper the bondsman was not intelligent and only increased the bounty by one unit per timestep and made all of the tasks either exclusive or not based on the experiment?s settings.  The goal of improving the bondsman is to create a task allocation method which will perform consistently and efficiently in both noisy and common scenarios.

My current proposal is to attempt a few variations of the bondsman due to the various inputs and decisions available to the bondsman.  I would like to look at creating a bondsman that adjusts the exclusivity of task classes based on either past knowledge or current system performance.  I may also use the bondsman's ability to control the functions for the bounties as another strategy for improving performance.  Since this is a multiagent system, I intend on analyzing the ramifications of the adaptive bondsman to strategies of the bounty hunters.

\section{Experiments \& Methodology}
I intend on continuing to use the same experiment testbed written in Java using the MASON simulation framework that was used in the original paper.  I will compare the bondsman methods I create for this project to the methods explored in the original paper in order to determine the success of the methods.

The measure for performance I will use is the sum total bounty over outstanding tasks at any given time.  For each method I will let the simulation run for 200,000 timesteps with 100 independent trials.  To compare the methods, I will take the average of the 100 trials at time step 200,000 and perform ANOVA with a tukey test (\(\alpha = .05\)).

\vspace{-0.5em}
\bibliographystyle{plain}
\bibliography{proposal}

\end{document}
